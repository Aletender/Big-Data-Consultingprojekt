\section{Einleitung}
\subsection{Problemstellung}
Im Rahmen der Master-Studiengangveranstaltung „Big Data Consulting Project“ wurde an die Studierenden des Master-Studienganges Big Data \& Business Analytics durch die DISH Consulting GmbH, welche Teil der Metro AG ist, eine Problemstellung im Rahmen der Gastronomie und Generativer künstlicher Intelligenz herangetragen.
Genauer gesagt beschäftigt sich die Problemstellung damit, wie Gastronomen an den Einsatz von Künstlicher Intelligenz herangeführt werden könnten, damit sie ihre betrieblichen Abläufe optimieren, bzw. revolutionieren können.
Im Detail strecken sich die Rahmenbedingungen der Problemstellung über die Betrachtung von Generativer Künstlicher Intelligenz Microservices und dem Einsatz der Google Cloud Plattform, da diese die etablierte Datenplattform der DISH Consulting GmbH ist.
Eine grundlegende Überlegung, welche an die Problemstellung dieser Projektarbeit gebunden ist, ist die Überlegung, dass der Klein- und Mittelständler Gastronom nicht über die finanziellen Mittel und zeitlichen Kapazitäten, als auch dem Wissen besitzt für ein erfolgreiches Social Media Marketing, um sein Lokal angemessen zu bewerben.
Social Media ist aus der heutigen Zeit kaum wegzudenken aus unternehmerischer Sicht, da diese Plattformen, je nach Plattform, einen Zugang zu einem Milliardenpublikum bietet.
Hier können Unternehmen mit gezielten Werbekampagnen ihren Umsatz ankurbeln.
Gleichzeitig sind mit den letzten Jahren viele Technologien im Bereich der Generativen Künstlichen Intelligenz etabliert worden, welche sich sehr gut in der Generierung von Texten und Bildern eignen.
Dementsprechend ergebe sich eine konkretere Problemstellung, die im Rahmen dieser Ausarbeitung behandelt wird, wie Gastronomen Generative Künstliche Intelligenz, in Form von Large Language Modellen, nutzen können, um ihre Social Media Präsenz zu etablieren, bzw. zu optimieren.
Durch eine Etablierte Social Media Präsenz könnten Gastronomen im Klein- und Mittelstand eine viel größere Zielgruppe ansprechen und folglich auch ihren Umsatz steigern.

\subsection{Methodik}
Basierend auf der oben beschriebenen Problemstellung, dass Gastronomen im Klein- und Mittelstand nicht über die Fähigkeiten besitzen Social Media Kampagnen zu generieren, welche zu einer höheren Aufmerksamkeit ihres Lokals führen würden und folglich auch zu mehr Umsatz, besteht das Ziel dieser Ausarbeitung ein Frontendtool zu entwickeln, welches basierend auf einem Generativen Künstlichen Intelligenz Microservice, dass Klein- und Mittelstand Gastronomen dabei helfen soll ihre Social Media Präsenz zu etablieren und zu optimieren.
Die Annahme dieses Ziels ist es, dass eine ordentliche Etablierung der Social Media Präsenz von Klein- und Mittelstandgastronomen zu mehr Umsatz und Bekanntheit führen würde.

\subsection{Zielsetzung}
Die Ausarbeitung dieser Problemstellung basiert auf der Kombination von zwei fundierten wissenschaftlichen Methoden.
Eine der beiden wissenschaftlichen Methoden ist die systematische Literaturrecherche.
Diese beschreibt ein strukturiertes Vorgehen problemstellungsrelevante Literatur zu beschaffen.
Dabei werden zunächst mehrere Suchstrings definiert, die in diversen wissenschaftlichen Datenbanken eingesetzt werden können.
Bei der Definition der Suchstrings wird auf englische, als auch auf deutsche Strings gesetzt, sowie der Einsatz von logischen Operatoren.
Als nächster Bedarf es die Auswahl von ordentlichen wissenschaftlichen Datenbanken, wie z. B. der IEEE Xplore, Google Scholar, oder SpringerLink.
Ebenfalls wird die Auswahl von Ergebnissen definiert, sodass vorzugsweise Peer-Reviewed Paper betrachtet werden, die nicht älter als 10 Jahre alt sind.
Literatur in Form von Büchern ist für grundlegende theoretische Zusammenhänge ebenfalls in einem angemessenen Rahmen in Ordnung.
Ergänzt wird die systematische Literaturrecherche mit der Erweiterung der Schneeballmethode.
Diese Erweiterung beschreibt ein Vorgehen, in welchem die bezogene Literatur, der Literatur, durchsucht wird auf relevante Quellen.
Dies erfolgt in einem iterativen Verfahren so lange, bis dieses Vorgehen keine relevante Literatur mehr liefert.

Die zweite wissenschaftliche Methode, auf die im Rahmen der Ausarbeitung der Problemstellung gesetzt wird, ist das Design Thinking.
Das Design Thinking kommt exakt dann zum Tragen, wenn es um die Entwicklung einer nutzerorientierten Lösung geht.
Dabei werden in einem aufbauenden Prozess zunächst die Zielgruppenbedürfnisse ana-lysiert – gefolgt von einer Definition der Problemstellung.
In einem kreativen Lösungsansatz werden Ideen entwickelt, welche dabei helfen sollen, einen Prototypen zu erstellen Als letztes wird der konzipierte und entwickelte Prototyp getestet und im besten Fall mit den Nutzern evaluiert und ggf. in eine nächste Iteration der Entwicklung und Brainstorming übergeben.