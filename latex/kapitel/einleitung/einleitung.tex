\section{Einleitung}
\subsection{Problemstellung}
Im Modul „Big Data Consulting Project“ wurde an die Studierenden, im Rahmen des Master-Studienganges Big Data \& Business Analytics, durch die DISH Consulting GmbH als Teil der Metro AG, eine Problemstellung der Gastronomieindustrie im Forschungsgebiet der Generativen Künstlichen Intelligenz herangetragen.
Genauer gesagt, beschäftigt sich die Problemstellung damit, wie Gastronomen an den Einsatz von Künstlicher Intelligenz herangeführt werden, um betrieblichen Abläufe und Geschäftsmodelle zu optimieren.
Die Rahmenbedingungen der Problemstellung umfassen die Betrachtung von Generativer Künstlicher Intelligenz als Microservice und dem Einsatz der Google Cloud Plattform als Infrastruktur, da diese die etablierte Datenplattform der DISH Consulting GmbH ist.
Eine grundlegende Überlegung, welche an die Problemstellung dieser Projektarbeit gebunden ist, ist die Überlegung, dass der Klein- und Mittelständler Gastronom nicht über die finanziellen Mittel und zeitlichen Kapazitäten, als auch dem Wissen besitzt für ein erfolgreiches Social Media Marketing, um sein Lokal angemessen zu bewerben.
Social Media ist für viele Unternehmen essenzieller Bestandteil der Marketingstrategie, da je nach Plattform, Zugang zu einem Milliardenpublikum ermöglicht wird.
Dem Einsatz von gezielten Werbekampagnen auf Social Media steht häufig eine Steigerung des Umsatzes zufolge.
Gleichzeitig sind mit den letzten Jahren viele Technologien im Bereich der Generativen Künstlichen Intelligenz etabliert worden, welche sich sehr gut für der Generierung von Texten und Bildern eignen.
Daraus ergibt sich die im Rahmen dieser Ausarbeitung behandelte konkrete Problemstellung, wie Gastronomen Generative Künstliche Intelligenz nutzen können, um ihre Social Media Präsenz zu stärken.
Durch eine etablierte Social Media Präsenz sollen Gastronomen im Klein- und Mittelstand eine größere Zielgruppe erreichen und folglich den Umsatz steigern.

\subsection{Zielsetzung}
Basierend auf der der Ausgangssituation, dass Gastronomen im Klein- und Mittelstand häufig nicht über die Fähigkeiten besitzen, Social Media Kampagnen zu generieren, welche zu einer höheren Aufmerksamkeit ihres Lokals führen würden und folglich auch zu mehr Umsatz, besteht das Ziel dieser Ausarbeitung darin, eine KI-basiert Anwendung zur syntethischen Produktion von Social Media Content zu entwickeln, welche Unternehmen im Klein- und Mittelstand im Gastronomiesektor dabei helfen soll, ihre Social Media Präsenz zu etablieren und zu optimieren unter der Annahme, dass eine ordentliche Etablierung der Social Media Präsenz sowohl den Umsatz, als auch die Bekanntheit der Marke steigert.


\subsection{Methodik}
Die Ausarbeitung dieser Problemstellung basiert auf der Kombination von zwei fundierten wissenschaftlichen Methoden.

Zunächst wird der aktuelle Forschungsstand in dem untersuchten Themengebiet mittel einer systematischen Literaturrecherche untersucht.
Diese beschreibt ein strukturiertes Vorgehen zur Identifiaktion problemstellungsrelevanter Literatur.\footcite{brocke2015standing}
Dabei werden zunächst mehrere Suchstrings definiert, die in diversen, wissenschaftlichen Datenbanken eingesetzt werden können.
Bei der Definition der Suchstrings wird auf englische, als auch auf deutsche Strings gesetzt, sowie der Einsatz von logischen Operatoren.
Als nächstes Bedarf es die Auswahl von wissenschaftlichen Datenbanken, wie z. B. der IEEE Xplore, Google Scholar, oder SpringerLink.
Ebenfalls wird die Auswahl von Ergebnissen definiert, sodass vorzugsweise Peer-Reviewed Paper betrachtet werden, die nicht älter als 10 Jahre alt sind.
Literatur in Form von Büchern ist für grundlegende theoretische Zusammenhänge ebenfalls in einem angemessenen Rahmen heranzuziehen.\footcite{xiao2019guidance}
Ergänzt wird die systematische Literaturrecherche durch die Anwendung der Schneeballmethode im Kontext der Literaturrecherche.
Die Schneeballmethode beschreibt ein Vorgehen, bei dem das Literaturverzeichnis releveanter Literatur, nach weiter nutzbarer Literatur durchsucht wird.
Das beschriebene Verfahren wird so häufig wiederholt, bis keine relevante Literatur identifiziert werden kann.

Die zweite wissenschaftliche Methode, auf die im Rahmen der Ausarbeitung der Problemstellung gesetzt wird, ist das Design Thinking.
Das Design Thinking kommt zum Tragen, wenn es um die Entwicklung einer nutzerorientierten Lösung geht.
Dabei werden in einem aufbauenden Prozess zunächst die Zielgruppenbedürfnisse analysiert, gefolgt von einer Definition der Problemstellung.
In einem kreativen Lösungsansatz werden Ideen entwickelt, welche dabei helfen sollen einen Prototypen zu erstellen.
Als letztes wird der konzipierte und entwickelte Prototyp getestet und im besten Fall mit den Nutzern evaluiert und ggf. in eine nächste Iteration der Entwicklung und Brainstorming übergeben.\footcite{heller2020design}