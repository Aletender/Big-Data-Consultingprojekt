\newpage
\section{Beratender Teil} \label{sec:beratender-teil}

\subsection{Entwicklung eines Erstenwurfs für das Applikationsdesign}\label{subsec:entwicklung-eines-erstenwurfs-fuer-das-applikationsdesign}
In diesem Kapitel wird der Erstentwurf für das Applikationsdesign vorgestellt.
Der Erstentwurf beschreibt sowohl die Frontend- als auch die Backend-Architektur der Applikation.

\textbf{Frontend}
Das Frontend stellt den Interaktionsbereich für den Benutzer dar.
Entsprechend ist es wichtig, dass das Frontend übersichtlich und intuitiv gestaltet ist.
Dabei ist es entscheidend, die komplexen Zusammenhänge der Applikation und abgebildeten Prozesse so zu visualisieren, dass der Benutzer nicht überfordert wird.
Die gesamte User-Experience (UX) soll so gestaltet sein, dass der Benutzer durch die Applikation geführt wird und die Akzeptanz der Applikation erhöht wird.

Zusätzlich soll sich die Applikation an die Corporate Identity des Unternehmens anpassen.
Dies soll durch die Verwendung der Unternehmensfarben in der Applikation und des Logos erreicht werden.

Zur Umsetzung gilt folgendes zu beachten:
\begin{itemize}
    \item Das DISH POS typische Orange soll als primäre Farbe verwendet werden.
    \item Als sekundäre Farben sollen Grau- und Blautöne verwendet werden.
    \item Die Farben sollen so gewählt werden, dass sie die Aufmerksamkeit des Benutzers auf wichtige Elemente lenken.
    \item Unternehmens- und Applikationslogos müssen an strategischen Stellen platziert werden.
\end{itemize}

\textbf{Backend}
Ziel der Backend-Architektur ist es, die Datenhaltung und -verarbeitung zu gewährleisten und die benötigten Aufrufe zu anderen Komponenten zu ermöglichen.
Im Rahmen dieses Erstenwurfs wird ein singulärer Server angenommen, der die Datenhaltung und -verarbeitung übernimmt.
Dieser Server wird über eine REST-API mit dem Frontend kommunizieren und die benötigten Daten bereitstellen.
Die REST-API setzt sich aus verschiedenen Endpunkten zusammen, welche jeweils die verfügbaren Ressourcen und Aktionen beschreiben.
Folgende Ressourcen stellt der Service bereit:
\begin{itemize}
    \item Endpunkt 1
    \item Endpunkt 2
    \item Endpunkt 3
\end{itemize}

Die Datenhaltung wird über eine Datenbank realisiert, welche innerhalb des Google Cloud Services betrieben wird.
Folgende Daten werden in der Datenbank gespeichert:
\begin{itemize}
    \item Prompts
    \item Antworten
    \item User-Feedback
    \item Generierte Bilder
    \item Sonstige Metadaten
\end{itemize}

Zum Speichern der textbasierten Daten kann entschieden werden, ob Cloud Spanner\footcite{google_spanner} als relationale Datenbank oder Data Store\footcite{google_datastore} als NoSQL-Datenbank verwendet wird.

Die Untersuchung von Khan et al. zeigt, dass sich der NoSQL-Ansatz besser dazu eignet, erste Prompts und deren Antworten zu speichern.
Durch die schemalose Struktur von NoSQL-Datenbanken können neue Prompts und Antworten einfach hinzugefügt werden, ohne dass die Datenbankstruktur angepasst werden muss.
Darüber hinaus lassen sich diese Technologien gut in die Google Cloud Services integrieren.
Der Einsatz von einer relationalen Datenbank bietet sich in einem nachgelagerten Schritt an, in welchem die gesammelten Daten analysiert, strukturiert und ausgewertet werden sollen.\footcite{Khan2022SQL}

Entsprechend wird für den produktiven Einsatz des Prototypen Data Store betrachtet.
Zur lokalen Entwicklung wird entsprechend MongoDB eingesetzt, da dies der NoSQL Favorit aus der vorangegangenen Recherche ist.

Neben den textbasierten Daten müssen auch die generierten Bilder gespeichert werden.
Dadurch wird eine effiziente Speicherung und Verwaltung der generierten Bilder gewährleistet und ein wiederverwendbarer Zugriff auf die Bilder ermöglicht.
Zur Speicherung von Bildern wird Google Cloud Storage\footcite{google_storage} verwendet.

\subsection{Umsetzung des Applikationsdesigns in ein Prototypen}\label{subsec:umsetzung-des-applikationsdesigns-in-ein-prototypen}







