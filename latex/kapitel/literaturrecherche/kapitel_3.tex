%! Author = mariu
%! Date = 11.01.2025

% Preamble
\documentclass[11pt]{article}

% Packages
\usepackage{amsmath}
\usepackage{tipa}

% Document
\begin{document}

\section{Systematische Literaturrecherche} \label{sec:literaturrecherche}
\subsection{Forschungsgegenstand}

Ziel der systematischen Literaturrecherche ist es, eine Übersicht über den aktuellen Stand der Forschung zu erhalten und die Grundlage für die Konzeptionierung einer vollständigen KI-basierten Anwendung zu schaffen, welche Gastronomen dabei helfen soll, ihre Social Media Präsenz zu etablieren und durch Verwendung von generativen Machine Learning Algorithmen zu optimieren.
Durch die systematische Literaturrecherche sollen aktuelle Trends im Einsatz von KI-Technologien im Bereich der Gastronomie und des Social Media Marketings identifiziert werden, die anschließend in der Umsetzung der Projektarbeit berücksichtigt werden können.
Die Durchführung der systematischen Literaturrecherche basiert auf dem nach vom Brocke et al. beschriebenen Vorgehen.

\subsection{Konzeption der Recherche} \label{sec:literaturrecherche_orga}

Zur Bestimmung der eingesetzten Suchstrings, der für die Identifikation relevanter Literatur in den Literaturdatenbanken benötigt wird, wird zunächst die Problemstellung in einzelne Keywords zerlegt.
Die im Rahmen dieser Projektarbeit behandelte Problemstellung ist die Nutzung von Generativen Künstlichen Intelligenz Modellen, um die Social Media Präsenz von Gastronomen zu etablieren und zu optimieren.
Daraus resultieren folgende Keywords:

\begin{itemize}
    \item Generative Künstliche Intelligenz
    \item Social Media
    \item Marketing
\end{itemize}

Die Verwendung dieser Keywords bietet die Grundlage zur inhaltlichen Filterung der Literaturdatenbanken nach inhaltlichen relevanten Publikationen, die sich mit der untersuchten Problemstellung auseinandersetzen.
Zur Entwicklung eines Suchstrings werden die Keywords mit logischen Operatoren wie AND und OR kombiniert, um die Suche zu verfeinern und die Suchergebnisse zu reduzieren.
Daraus ergab sich der folgende primäre Suchstring, welcher in den Literaturdatenbanken eingesetzt wird:

- ("Generative AI" OR "Artificial Intelligence" OR "Generative Models") AND ("AI-Driven Marketing" OR "Marketing Applications" OR "Content Creation" OR "Social Media Advertising")

Die Durchführung der systematischen Literraturrecherche wird auf den folgenden online Portalen unter Verwendung des VPN-Zugangs der Hochschule durchgeführt:

- EBSCO Discovery Service: https://eds.b.ebscohost.com
- Elsevier ScienceDirect: http://www.sciencedirect.com
- Emerald Insight: https://www.emeraldinsight.com
- Google Scholar: https://scholar.google.de
- IEEE Xplore: https://ieeexplore.ieee.org
- Springer Link: https://link.springer.com

Die Suchergebnisse werden nach Relevanz und Aktualität gefiltert, wobei bevorzugt Peer-Reviewed Paper betrachtet werden, die nicht älter als 10 Jahre sind.
Für die Finale Auswahl einer Quelle erfolgt eine individuelle Bewertung anhand der Prüfung nach inhaltlicher Relevanz durch Untersuchung des Abstracts sowie des Inhaltsverzeichnisses und zudem der Qualtität der Quelle durch ermittlung des H-Indexes.
Dokumentiert werden die Suchergebnisse in einer Literaturliste nach folgendem Schema:

\begin{tabular}{|l|l|l|l|l|l|}
\hline
Suchort & Suchalgorithmus & Anzahl Treffer & Auswahl & H-Index & Einschränkungen \\ \hline
\end{tabular}

Da die Suchergebnisse in den Literaturdatenbanken zu viele Treffer liefern, wird die Suche auf die letzten 10 Jahre beschränkt, um die Anzahl der Treffer zu reduzieren.
Zudem erfolgt eine Modifizierung des Suchstrings um den Begriff \("\)Gastronomie\("\), um die Literaturrecherche auf den Erkenntnissen im Anwendungsbereich der Gastronomie einzuschränken.

\subsection{Durchführungen der Recherche}

Die Durchführung der Literaturrecherche folgt dem in Kapitel \ref{sec:literaturrecherche_orga} beschriebenen Vorgehen.
Die Suchstrings werden in den genannten Datenbanken eingesetzt, um relevante Literatur zu identifizieren.

\subsection{Auswahl der Literatur}
Nach erfolgreicher Durchführung der Recherche werden die Treffer gesichtet und die relevanten Publikationen ausgewählt.
Die auswahl an relevanter Literatur erfolgt unter Berücksichtigung des Publikationsdatums und der inhaltlichen Relevanz für die Problemstellung, sowie der Qualität der Quelle durch Untersuchung des H-Index und der Anzahl der Zitationen der Quelle.
Die als relevant erachteten Publikationen werden in eine Literaturliste aufgenommen und tabellarisch erfasst.

\subsection{Analyse und Synthese der Literatur}
Abschließend werden die ausgewählten Publikationen analysiert und die Ergebnisse in einer Konzeptmatrix nach Webster und Watson zusammengefasst.

\end{document}