%! Author = mariu
%! Date = 11.01.2025

% Preamble
\documentclass[11pt]{article}

% Packages
\usepackage{amsmath}
\usepackage{tipa}

% Document
\begin{document}

\section{Systematische Literaturrecherche} \label{sec:literaturrecherche}
\subsection{Forschungsgegenstand}

Ziel der systematischen Literaturrecherche ist es, eine Übersicht über den aktuellen Stand der Forschung zu erhalten und die Grundlage für die Konzeptionierung einer vollständigen KI-basierten Anwendung zu schaffen, welche Gastronomen dabei helfen soll, ihre Social Media Präsenz zu etablieren und durch Verwendung von generativen Machine Learning Algorithmen zu optimieren.
Durch die systematische Literaturrecherche sollen aktuelle Trends im Einsatz von KI-Technologien im Bereich der Gastronomie und des Social Media Marketings identifiziert werden, die anschließend in der Umsetzung der Projektarbeit berücksichtigt werden können.

\subsection{Konzeption der Recherche} \label{sec:literaturrecherche_orga}

Zur Bestimmung der eingesetzten Suchstrings, der für die Identifikation relevanter Literatur in den Literaturdatenbanken benötigt wird, wird zunächst die Problemstellung in einzelne Keywords zerlegt.
Die im Rahmen dieser Projektarbeit behandelte Problemstellung ist die Nutzung von Generativen Künstlichen Intelligenz Modellen, um die Social Media Präsenz von Gastronomen zu etablieren und zu optimieren.
Daraus resultieren folgende Keywords: Generative Künstliche Intelligenz, Social Media, Gastronomie.
Die Suchstrings mit den entsprechenden Keywords werden in englischer und deutscher Sprache formuliert und mit logischen Operatoren kombiniert, um die Suche zu verfeinern.
Die untersuchten Datenbanken sind IEEE Xplore, Google Scholar und SpringerLink, EBSCO und ScienceDirect.

\subsection{Durchführungen der Recherche}

Die Durchführung der Literaturrecherche folgt dem in Kapitel \ref{sec:literaturrecherche_orga} beschriebenen Vorgehen.
Die Suchstrings werden in den genannten Datenbanken eingesetzt, um relevante Literatur zu identifizieren.

\subsection{Auswahl der Literatur}
Nach erfolgreicher Durchführung der Recherche werden die Treffer gesichtet und die relevanten Publikationen ausgewählt.
Die auswahl an relevanter Literatur erfolgt unter Berücksichtigung des Publikationsdatums und der inhaltlichen Relevanz für die Problemstellung, sowie der Qualität der Quelle durch Untersuchung des H-Index und der Anzahl der Zitationen der Quelle.
Die als relevant erachteten Publikationen werden in eine Literaturliste aufgenommen und tabellarisch erfasst.

\subsection{Analyse und Synthese der Literatur}
Abschließend werden die ausgewählten Publikationen analysiert und die Ergebnisse in einer Konzeptmatrix nach Webster und Watson zusammengefasst.

\end{document}