\newpage
\section{Wissenschaftlicher Teil}

\subsection{Analyse des Forschungsstandes}

\subsection{Strategische Geschäftsmodellentwicklung}

\subsection{Evaluation einzusetzender Technologien}

\subsubsection{Auswahlprozess Web-Technologien}
Um geeignete Technologien für die Entwicklung der Applikation zu finden, wird ein Auswahlprozess durchgeführt.

Dieser Prozess folgt dem Konzept der PAPRIKA Methode von Hansen und Ombler.\footcite{Paprika2008}
Zu Beginn wird eine Liste von Kriterien erstellt, die für die Auswahl der Technologien relevant sind.
Anschließend wird eine Gewichtung der Kriterien vorgenommen, um die Relevanz der einzelnen Kriterien zu bestimmen.
Folgende Anforderungen wurden identifiziert:

\begin{table}[htbp]
  \centering
  \begin{tabular}{|p{2cm}|c|p{5cm}|p{4cm}|}
      \hline
      \textbf{Kriterium} & \textbf{Gewicht} & \textbf{Beschreibung} & \textbf{Skala}\\ \hline
      {Lernkurve} & 0.4 & Lernaufwand in Relation zum Erfolgsgrad im Hinblick auf umzusetzende Features & flach, moderat, steil\\ \hline
      {Community} & 0.3 & Größe und Aktivität der Community sowie vorhandenes Lernmaterial & klein, mittel, groß\\ \hline
      {Bibliotheken} & 0.2 & Verfügbarkeit von Bibliotheken & begrenzt, mittel, umfangreich\\ \hline
      {Relevanz} & 0.1 & Aktualität und Weiterentwicklung der Technologie & niedrig, mittel, hoch\\ \hline
  \end{tabular}
  \caption{Bewertung der Anforderungen an Web-Technologien}\label{tab:table}
\end{table}

Im nächsten Schritt werden die zu vergleichenden Technologien identifiziert.
Folgende Komponenten werden für die Entwicklung der Applikation benötigt:
\begin{itemize}
  \item Frontend-Framework
  \item Backend-Framework
  \item Datenbank
\end{itemize}

Um den Auswahlprozess zu vereinfachen, wird die Auswahl auf drei Technologien je Kategorie beschränkt.
Dabei erfolgt die Auswahl anhand von bereits vorhandenem Wissen und Erfahrungswerten.
\begin{table}[htbp]
  \centering
  \begin{tabular}{|l|c|c|c|}
      \hline
      \textbf{Technologie Kategorie} & \textbf{Top 1} & \textbf{Top 2} & \textbf{Top 3} \\ \hline
      {Frontend-Framework} & Angular & VueJS & React \\ \hline
      {Backend-Framework} & Flask & Django & Cherrypy \\ \hline
      {Datenbank} & PostgreSQL & MySQL & MongoDB \\ \hline
  \end{tabular}
  \caption{Technologieauswahl Übersicht}\label{tab:Technologieauswahl Übersicht}
\end{table}

Zur Bestimmung der einzelnen Werte werden unterschiedliche Datenquellen herangezogen.
Zur Überprüfung der Relevanz wird die aktuelle Stackoverflow Developer Survey 2024\footcite{StackOverflow2024}, sowie State of JS 2023\footcite{stateofjsStateJavaScript2023} herangezogen.

Zusätzlich werden die Plattformen Github.com und Stackoverflow.com untersucht, inwiefern die Technologien dort vertreten sind.
Für die Frontend Frameworks im speziellen werden die verfügbaren Libraries im Node Package Manager (NPM) untersucht.

Aus der Analyse lassen sich folgende Daten ableiten:

\begin{table}[h!]
    \centering
    \begin{tabular}{|l|p{2cm}|p{3cm}|p{3cm}|p{3cm}|}
        \hline
        \rowcolor{lightgray} Name & \textbf{Datum Veröffentlichung} & \textbf{Aktive Fragen auf Stackoverflow} & \textbf{Repositories auf Github mit Tag} & \textbf{Abhängigkeiten NPM} \\ \hline
        Angular & 2010\footcite{githubAngularReleaseV090} & 306.845 & 57.588 & 14.607 \\ \hline
        VueJS & 2014\footcite{eggheadEvanYou} & 108.341 & 26.600 & 80.824 \\ \hline
        React & 2013\footcite{githubReactCHANGELOG} & 481.823 & 173.000 & 240.000 \\ \hline
        \hline
        Flask & 2010\footcite{pypiFlask} & 55.856 & 50.985 & - \\ \hline
        Django & 2005\footcite{pypiDjango} & 313.041 & 67.366 & - \\ \hline
        Cherrypy & 2004\footcite{pypiCherryPy} & 1.370 & 147 & - \\ \hline
        \hline
        PostgreSQL & 1996\footcite{postgresqlHappyBirthday} & 178.607 & 56.562 & - \\ \hline
        MySQL & 1995\footcite{amazonWhatMySQL} & 661.661 & 75.826 & - \\ \hline
        Mongodb & 2009\footcite{mongodbMongoDBEvolved} & 176.192 & 111.693 & - \\ \hline
    \end{tabular}
    \caption{Bewertung von Technologien}\label{tab:Analyseergebnise Relevanz der Plattformen}
\end{table}

Auf Basis der gesammelten Daten und geleisteten Recherchen lassen sich die Technologien wie folgt bewerten:

\begin{table}[h!]
    \centering
    \begin{tabular}{|l|l|c|c|c|c|}
        \hline
        \rowcolor{lightgray} \textbf{Technologie} & \textbf{Kategorie} & \textbf{Community} & \textbf{Lernkurve} & \textbf{Bibliotheken} & \textbf{Relevanz} \\ \hline
        Angular & Backend Framework & \cellcolor{green!70}3 & \cellcolor{red!70}1 & \cellcolor{green!70}3 & \cellcolor{orange!70}2 \\ \hline
        VueJS & Backend Framework & \cellcolor{green!70}3 & \cellcolor{green!70}3 & \cellcolor{orange!70}2 & \cellcolor{green!70}3 \\ \hline
        React & Backend Framework & \cellcolor{green!70}3 & \cellcolor{orange!70}2 & \cellcolor{green!70}3 & \cellcolor{green!70}3 \\ \hline
        Flask & Frontend Framework & \cellcolor{orange!70}2 & \cellcolor{green!70}3 & \cellcolor{orange!70}2 & \cellcolor{orange!70}2 \\ \hline
        Django & Frontend Framework & \cellcolor{green!70}3 & \cellcolor{red!70}1 & \cellcolor{green!70}3 & \cellcolor{green!70}3 \\ \hline
        Cherrypy & Frontend Framework & \cellcolor{red!70}1 & \cellcolor{orange!70}2 & \cellcolor{red!70}1 & \cellcolor{red!70}1 \\ \hline
        PostgreSQL & Datenbank & \cellcolor{green!70}3 & \cellcolor{orange!70}2 & \cellcolor{green!70}3 & \cellcolor{green!70}3 \\ \hline
        MySQL & Datenbank & \cellcolor{green!70}3 & \cellcolor{green!70}3 & \cellcolor{orange!70}2 & \cellcolor{orange!70}2 \\ \hline
        Mongodb & Datenbank & \cellcolor{green!70}3 & \cellcolor{orange!70}2 & \cellcolor{green!70}3 & \cellcolor{green!70}3 \\ \hline
    \end{tabular}
    \caption{Bewertung von Technologien}\label{tab:Tabllee Bewertung von Technologien}
\end{table}
